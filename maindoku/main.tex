\documentclass{fancydocument}

\usepackage{enumerate}

\title{IDPA: Programmiersprachen benchmarken}
\author{Nick Zbinden und Matthias Gasser}
\date{\today}
\begin{document}
\maketitle
\thispagestyle{fancy}

\section{Vorwort}

Dieses Projrkt ist aus dem Willen heraus entstanden eine Arbeit zu
machen die Applikations. und Systemtechnikenkenntniss nuetzt und
verbindet. Nach verschiedenen Ideen einigten wir uns ... 

\section{Abstract}
\section{Einleitung}
\section{Material und Methoden}
\subsection{Allgemeines zu den Versuchen}
\subsection{Verwendete Programmiersprachen}
\subsection{Das Testsystem}
\subsection{Aufnahme der Daten}
\subsection{Datenauswertungsmethoden}
\section{Ergebnisse}

\subsection{Allgemeines zu den Sprachen}

Um es einfacher und fair zu machen haben wir uns Entschieden
Programmiersprachen zu nehmen die auf der JVM lafen

\subsection{Java}
\subsection{Scala}
\subsection{Clojure}

Clojure ist eine dynamische Programmiersprache 

\section{Diskussion}
\subsection{Vergleich der Resultate}
\subsection{Schlussfolgerung}
\section{Abk\"urzungsverzeichnis}

VM  - Virtuall Maschine
JVM - Java Virtuell Maschine
IL - Interniediet Language

\section{Literaturverzeichnis}
\section{Glossar}

Virtuall Maschine: Eine Virtuall Maschine ist eine Programm welches simuliert eine echte
Maschine zu sein. Sie bekommt als input irgend eine Sprache und fuert
diese auf der darunter liegenden Sprache aus.

Java Virtuall Maschiene: Eine implementation einer VM die dafuer ausgelegt ist Java Bytecode auszufueren.
Bytecode

Java Bytecode: The Java Bytecode wurde als Zielsprache fuer 

\section{Anhang}


\end{document}
